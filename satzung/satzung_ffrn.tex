\documentclass[12pt,a4paper,titlepage]{scrartcl}
\usepackage[utf8]{inputenc}
\usepackage[german]{babel}
\usepackage{amsmath}
\usepackage{amsfonts}
\usepackage{amssymb}
\usepackage{enumerate}
\usepackage{listings}
\usepackage{url}
\usepackage{MnSymbol}
\usepackage{color}
\definecolor{grey}{rgb}{0.9,0.9,0.9}


\usepackage{graphicx}               % erlaubt einfügen von Bildern
\usepackage{float}                  %Bilder Fixieren                            
\usepackage{caption}                % Für Captions in Minipage (für bilder alternativ zu \caption ->\captionof verwenden
\graphicspath{{./images/}}          % suche nach Bildern im angegebenen Ordner
\usepackage[printonlyused]{acronym} %ermöglicht das Arbeiten mit Abk. und Verzeichniss
\usepackage{placeins}               % für \FloatBarrier damit figures auch wirklich genau da stehen bleiben so sie sind... Nicht schön aber ich glaub der Holzhammer muss hier sein... 
\usepackage{cite}                                                               
\def\BibTeX{{\rm B\kern-.05em{\sc i\kern-.025em b}\kern-.08em                   
   T\kern-.1667em\lower.7ex\hbox{E}\kern-.125emX}}
\date{08. Juli 2018}   
\title{Satzung Freifunk Rhein Neckar e.V.}
\lstset{prebreak=\raisebox{0ex}[0ex][0ex]
        {\ensuremath{\rhookswarrow}}}
\lstset{postbreak=\raisebox{0ex}[0ex][0ex]
        {\ensuremath{\rcurvearrowse\space}}}
\lstset{
	breaklines=true,
	breakatwhitespace=false,
	backgroundcolor=\color{grey},
}
\renewcommand*\thesection{\S~\arabic{section}}
\begin{document}
\maketitle
\pagenumbering{Roman}
\thispagestyle{empty}
\newpage
\pagenumbering{arabic}
\setcounter{page}{1}

\section{Name und Sitz des Vereins}
\begin{enumerate}
\item Der Verein führt den Namen „Freifunk Rhein-Neckar“ ( im folgenden Verein genannt).
\item Der Sitz des Vereines ist Mannheim.
\item Der Verein ist in das Vereinsregister der Stadt Mannheim einzutragen und trägt 
danach den Namen „Freifunk Rhein-Neckar e.V.“.
\item Das Geschäftsjahr des Vereins ist das Kalenderjahr. 
\end{enumerate}

\section{Zweck des Vereins, Gemeinnützigkeit, Auflösung und Vermögen.}
\begin{enumerate}
\item Zweck des Vereins ist die Erforschung, Anwendung und Verbreitung freier 
Netzwerktechnologien sowie die Verbreitung und Vermittlung von Wissen über Funk und 
Netzwerktechnologien. 
\item Der Satzungszweck wird insbesondere verwirklicht durch:
\begin{itemize}
\item die Förderung der Bildung und Forschung bezüglich moderner Kommunikationsnetze, insbesondere durch das Internet und durch Vorträge, Veranstaltungen, Vorführungen und Publikationen
\item der Förderung und Unterstützung des Zugangs zu Informationstechnologie für sozial benachteiligte Personen 
\item  der Schaffung experimenteller Kommunikations- und Infrastrukturen sowie 
Bürgerdatennetzen. 
\item Kulturelle, technologische und soziale Bildungs- und Forschungsobjekte 
\item die Veranstaltung regionaler, nationaler und internationaler Kongresse, Treffen 
und Konferenzen, sowie die Teilnahme der Mitglieder. 
\end{itemize}
\item Der Verein ist frei und unabhängig. Er verfolgt ausschließlich und unmittelbar 
gemeinnützige Zwecke im Sinne des Abschnitts „Steuerbegünstigte Zwecke“ der 
Abgabenordnung. Der Verein agiert unter Ausschluss von parteipolitischen und konfessionellen Gesichtspunkten. Er ist selbstlos tätig und verfolgt nicht in erster Linie eigenwirtschaftliche Zwecke. Die Mittel des Vereins dürfen nur für satzungsgemäße 
Zwecke verwendet werden. Es darf keine Person durch Ausgaben, die dem 
Vereinszweck fremd sind, oder durch unverhältnis hohe Vergütungen begünstigt 
werden. Die Mitglieder erhalten keine Zuwendungen aus den Mitteln des Vereins. 
\item Bei Auflösung der Körperschaft oder bei Wegfall steuerbegünstigter Zwecke fällt das Vermögen des Vereins entsprechend vorheriger Festlegung durch die Mitgliederversammlung ganz oder in Teilen an einen oder mehrere der Folgenden Vereine:
\begin{itemize}
\item den RaumZeitLabor e.V

\item MRMCD e.V.

\item den Freifunk Rheinland e.V.

\item an eine andere steuerbegünstigte Körperschaft oder Körperschaft öffentlichen Rechts welcher es unmittelbar für gemeinnützige Zwecke im Bereich der Bildung verwenden darf.
\end{itemize}
\item Ausscheidende Mitglieder haben keinen Anspruch auf das Vereinsvermögen. 
\item Über die Auflösung des Vereines entscheidet eine Mitgliederversammlung, die eigens 
zu diesem Zweck einberufen wird. Die Auflösung gilt als beschlossen wenn 75\% 
der abgegebenen Stimmen dafür stimmen.
\end{enumerate}


\section{Mitgliedschaft}
Mitglieder können natürliche und juristische Personen, z. B. Firmen, Vereine, 
Verbände und Behörden werden, die gewillt sind, die gemeinnützigen Ziele des Vereins 
zu fördern und diesen in der Durchführung seiner Aufgaben zu unterstützen. 
Körperschaften, Vereine und Verbände können die Mitgliedschaft entweder nur für sich 
selbst oder auch für ihre Mitglieder erwerben. Bei Minderjährigen ist die Zustimmung des 
gesetzlichen Vertreters erforderlich.
\begin{enumerate}
\item Der Aufnahmeantrag ist schriftlich, auch in elektronischer Form, an den 
Vorstand zu richten, der über die Aufnahme des Antragstellers entscheidet. 
\item Das aufgenommene Mitglied erhält eine Kopie der Satzung. Die jeweils 
aktuelle Satzung wird darüber hinaus an geeigneter Stelle den Mitgliedern 
verfügbar gemacht. 
\item Der Beitritt gilt erst dann als vollzogen, wenn der Mitgliedsbeitrag entrichtet 
worden ist. 
\item Die Mitglieder haben das Recht, an der Mitgliederversammlung des Vereins 
teilzunehmen, Anträge zu stellen, und das Stimmrecht auszuüben. Juristische 
Personen üben ihr Stimmrecht durch bevollmächtigte Vertreter aus. 
\item Jedes Mitglied hat einen Jahresbeitrag zu leisten, dessen Höhe und Fälligkeit 
in der Finanzordnung festgehalten ist. Diese wird von der 
Mitgliederversammlung beschlossen. 
\item Der Vorstand kann der Mitgliederversammlung die Ernennung von 
Ehrenmitgliedern vorschlagen. Ehrenmitglieder sind von der Beitragszahlung
freigestellt und haben auf Mitgliederversammlungen volles Stimmrecht. 
\item Auf Antrag kann der Vorstand Mitgliedsbeiträge stunden und ganz oder 
teilweise erlassen. 
\item Die Mitgliedschaft endet durch Austritt, Ausschluss oder Tod. 
\item Der Austritt muss durch schriftliche Mitteilung an den Vorstand erklärt werden. 
Er wird mit Endes des Geschäftsjahrs wirksam und muss sechs Wochen vor 
dessen Ablauf mitgeteilt worden sein. Auf Wunsch des Mitglieds kann die 
Wirksamkeit auch mit sofortiger Wirkung eintreten. 
\item Der Ausschluss eines Mitgliedes erfolgt durch den Vorstand, wenn dieses gegen die Satzungsbestimmungen, die sich daraus ergebenden Pflichten oder in sonstiger Weise gegen die Interessen des Vereins verstößt. Der Ausgeschlossene kann 
innerhalb eines Monats nach Zugang des Beschlusses Einspruch einlegen und 
die nächste Mitgliederversammlung anrufen, von der die Gültigkeit des 
Ausschlusses mit Dreiviertelmehrheit der anwesenden Mitglieder bestätigt oder 
der Ausschluss rückgängig gemacht werden kann. Vom Zeitpunkt des 
Einspruchs bis zur Entscheidung über den Ausschluss besteht die Mitgliedschaft 
weiter. 
\end{enumerate}

\section{Organe des Vereins}
\begin{enumerate}[I.]
\item Die Mitgliederversammlung 
	\begin{enumerate}[1.]
	\item Die ordentliche Mitgliederversammlung findet einmal jährlich statt. 
	\item Der Vorstand hat eine außerordentliche Mitgliederversammlung unverzüglich und 
	unter genauer Angabe von Gründen einzuberufen, wenn es das Interesse des Vereins 
	erfordert oder wenn mindestens 10 Prozent der Mitglieder dies schriftlich unter Angabe des 
	Zwecks und der Gründe vom Vorstand verlangen.
	\item Die Leitung der Versammlung hat ein Mitglied des Vorstands oder ein von der 
	Mitgliederversammlung bestimmter Versammlungsleiter. 
	\item Die Beschlüsse der Mitgliederversammlung werden in einem Protokoll niedergelegt 
	und mit den Unterschriften des Versammlungsleiters und des Protokollführers 
	beurkundet. 
	\item Der Mitgliederversammlung obliegen:
		\begin{enumerate}[a)]
		\item Beschlussfassung über alle den Verein betreffenden Angelegenheiten von 
		grundsätzlicher Bedeutung, 
		\item Entscheidung über fristgemäß eingebrachte Anträge. 
		\item Entgegennahme des Jahresberichtes des Vorstands.
		\item Entlastung des Vorstands.
		\item Wahl der Vorstandsmitglieder. 
		\item Beschlussfassung über Satzungsänderungen. 
		\item Festsetzung der Mitgliedsbeiträge.
		\item Die Auflösung des Vereins gemäß § 2, Ziffer 4 und 6 dieser Satzung.
		\end{enumerate}
 
	\item Fristen:
		\begin{enumerate}[a)]
		\item Die Versammlung wird mindestens vier Wochen vor dem 
		Versammlungstermin mit einer schriftlichen Mitteilung (per Post oder per E-Mail) an die Mitglieder 
		angekündigt.
		\item Ein Antrag an die Mitgliederversammlung gilt als fristgemäß eingereicht, wenn 
		er zwei Wochen vor Beginn der Mitgliederversammlung beim Vorstand 
		eingegangen ist.
		\end{enumerate}
	\end{enumerate}
\item Der Vorstand
	\begin{enumerate}[1.]
	\item Der Vorstand ist für alle laufenden Angelegenheiten des Vereins und seine 			Vertretung nach außen verantwortlich.
	\item Der Vorstand besteht aus 3 Personen von denen eine von der Mitgliederversammlung mit der hauptamtlichen Finanzverwaltung des Vereins beauftragt wird.
	\item Der Vorstand wird von der Mitgliederversammlung für die Dauer von 2 Jahren gewählt.
	\item Jedes Vorstandsmitglied ist alleinvertretungsberechtigt.
	\end{enumerate}
\end{enumerate}


\section{Schlussbestimmung}
\begin{enumerate}[I.]
\item Der Vorstand ist befugt, redaktionelle Änderungen an dieser Satzung, sofern sie 
einer Auflage des Registergerichtes oder einer Behörde entsprechen müssen, 
durchzuführen.
\end{enumerate}
\end{document}
