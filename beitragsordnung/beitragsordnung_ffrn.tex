\documentclass[12pt,a4paper,titlepage]{scrartcl}
\usepackage[utf8]{inputenc}
\usepackage[german]{babel}
\usepackage{amsmath}
\usepackage{amsfonts}
\usepackage{amssymb}
\usepackage{enumerate}
\usepackage{listings}
\usepackage{url}
\usepackage{MnSymbol}
\usepackage{color}
\definecolor{grey}{rgb}{0.9,0.9,0.9}


\usepackage{graphicx}               % erlaubt einfügen von Bildern
\usepackage{float}                  %Bilder Fixieren                            
\usepackage{caption}                % Für Captions in Minipage (für bilder alternativ zu \caption ->\captionof verwenden
\graphicspath{{./images/}}          % suche nach Bildern im angegebenen Ordner
\usepackage[printonlyused]{acronym} %ermöglicht das Arbeiten mit Abk. und Verzeichniss
\usepackage{placeins}               % für \FloatBarrier damit figures auch wirklich genau da stehen bleiben so sie sind... Nicht schön aber ich glaub der Holzhammer muss hier sein... 
\usepackage{cite}                                                               
\def\BibTeX{{\rm B\kern-.05em{\sc i\kern-.025em b}\kern-.08em                   
   T\kern-.1667em\lower.7ex\hbox{E}\kern-.125emX}}
   
\title{Finanz- und Beitragsordnung Freifunk Rhein Neckar e.V.}
\lstset{prebreak=\raisebox{0ex}[0ex][0ex]
        {\ensuremath{\rhookswarrow}}}
\lstset{postbreak=\raisebox{0ex}[0ex][0ex]
        {\ensuremath{\rcurvearrowse\space}}}
\lstset{
	breaklines=true,
	breakatwhitespace=false,
	backgroundcolor=\color{grey},
}
\renewcommand*\thesection{\S~\arabic{section}}
\begin{document}
\maketitle
\pagenumbering{Roman}
\thispagestyle{empty}
\newpage
\pagenumbering{arabic}
\setcounter{page}{1}

\section{Grundsätze}
\begin{enumerate}
\item Grundlagen dieser Finanzordnung sind:
	\begin{enumerate}
	\item Die Satzung des Freifunk Rhein-Neckar e.V.
	\item Die Beschlüsse der Mitgliederversammlung 
	\item Mittel des Vereins dürfen nur für im Sinne der Satzung und dieser Finanzordnung 
verwendet werden. 
	\item Haushaltsjahr ist das Kalenderjahr 
		\begin{enumerate}
		\item Das Kalenderjahr 2014 ist ein Rumpfgeschäftsjahr und beginnt mit der Gründung. 
		\end{enumerate}
	\end{enumerate}
\end{enumerate}
	
\section{Verantwortlichkeiten}
\begin{enumerate}
\item Verantwortlich für die finanzielle Tätigkeit des Vereins ist der Schatzmeister. 
\item Berichterstattung 
	\begin{enumerate}
	\item Im Rahmen der Vorstandssitzungen erstattet der Schatzmeister Bericht über 
die aktuelle finanzielle Situation des Vereins. 
	\item Der Finanzbericht ist zur ersten Mitgliederversammlung im Folgejahr durch 
den Schatzmeister vorzulegen. 
	\item Jeder, der im Namen des Vereins Gelder einnimmt oder ausgibt, hat dies ordentlich zu dokumentieren. Hierzu gehören:
		\begin{enumerate}
		\item Datum 
		\item Art der Einnahme/Ausgabe 
		\item Betrag 
		\item Auslagen werden nur gegen Einreichung von Belegen erstattet. 
		\end{enumerate}
	\end{enumerate}
\end{enumerate}

\section{Einnahmen}
\subsection{Mitgliedsbeiträge}
\begin{enumerate}
\item Der Mitgliedsbeitrag beträgt:
	\begin{enumerate}
	\item für jedes ordentliche Mitglied 60€ .
	\item Mitglieder, die einen höheren Beitrag zahlen, erwerben mit Zahlung den Status 
„Förderndes Mitglied“.
	\item Jedes Mitglied hat das Recht einen Antrag auf verminderten Mitgliedsbeitrag in 
Höhe von 12€ zu stellen. Der Vorstand entscheidet über jeden Antrag im 
Einzelfall. 
	\item Der Jahresbeitrag zum 1. Januar eines jeden Jahres fällig. 
	\item Der Mitgliedsbeitrag für das laufende Jahr wird anteilig für jeden noch nicht 
angefangenen Monat berechnet.   
	\end{enumerate}
\item Beitragsrückstand
	\begin{enumerate}
	\item Gerät ein Mitglied mit der Zahlung des Jahresbeitrags in Rückstand 
wird ab dem folgenden Monat das Mahnverfahren angestoßen. Erfolgt 
auch auf diese Mahnung kein Zahlungseingang innerhalb von sechs 
Wochen ist der Vorstand ermächtigt, den Ausschluss des Mitglieds zu 
beschließen.
	\end{enumerate}
\end{enumerate}
\subsection{Einnahmen im Rahmen von Veranstaltungen}
\begin{enumerate}
\item Einnahmen im Rahmen von Veranstaltungen sind gemäß der Finanzordnung zu 
dokumentieren. 	
\end{enumerate}

\subsection{Zuwendungen}
\begin{enumerate}
\item Zuwendende erhalten nach Anfertigung des Jahresabschlusses eine 
Zuwendungsbescheinigung. Diese kann auch auf Wunsch innerhalb von 14 Tagen nach 
Zuwendung per Post zugestellt werden.  
\end{enumerate}
\section{Ausgaben}
\begin{enumerate}
\item Zulässig sind:
	\begin{enumerate}
	\item Ausgaben im Sinne der Satzung.
	\item Kosten der laufenden Geschäftstätigkeit (z.B. Gebühren, Porto, Büromaterial, 
Postfach, Geschäftsstelle, Telefonkosten).
	\end{enumerate}
	\item Bis zu einer Höhe von 20€ ist jedes Vorstandsmitglied einzeln entscheidungsberechtigt. 
	\item Bis zu einer Höhe von 100€ ist der Vorstand mit einfacher Mehrheit entscheidungsberechtigt.
	\item Ab einer Höhe von 100€ muss der Vorstand einstimmig entscheiden.
	\item Bei Entscheidungen über die Förderung von Vereinsmitgliedern im Sinne satzungsgemäßer Zwecke haben die Nutznießer kein Stimmrecht. 
	\item Diese Festlegung gilt nur für die Beschlussfassung im Innenverhältnis. Die 
Handlungsbefugnis des Vereins im Außenverhältnis, insbesondere die Verfügung für 
Vereinskonten, ist davon nicht betroffen.
\end{enumerate}
 
\section{Inkrafttreten und Geltungsdauer }
\begin{enumerate}
\item Diese Finanzordnung gilt zeitlich unbegrenzt und kann nur durch die 
Mitgliederversammlung geändert werden. 
\item Redaktionelle Änderungen sind hiervon nicht betroffen. 
\end{enumerate}
\end{document}
